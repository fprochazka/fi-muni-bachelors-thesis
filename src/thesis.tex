\documentclass[%
  %% This option enables the default options for the digital version.
  %% Replace with `printed` to enable the default options for the printed version.
  digital,
  % Tato volba umožní sazbu práce pouze jednostraně,
  % je nastavena implicitně. Sazba je pouze na stranách lichých.
  % Tato volba je implicitní a doporučována.
  oneside,
  %% removes the black and white cover
  % nocover, % printed has no cover by default
  % Po nastavení této volby bude špatně zalomený text na koncích řádků
  % zvýrazněn černým obdélníčkem pro snažší vizuální identifikaci.
  % Dále volbu přebírají další balíky, jako je graphics,
  % a zde způsobí sazbu rámečků místo vkládání obrázků.
  % draft,
  %% This option enables colorful typesetting. Replace with
  %% `monochrome`, if you are going to print the thesis on a monochromatic printer.
  color,
  %% Causes the coloring of tables. Replace with `notable` to restore plain tables.
  table,
  %% Prints the List of Figures. Replace with `nolof` to hide the List of Figures.
  % lof,
  %% Prints the List of Tables. Replace with `nolot` to hide the List of Tables.
  nolot, % lot
  %% The microtypographic extension of modern TEX engines.
  %% microtype, % should be on by default
  %% More options are listed in the class documentation at
  %% <http://mirrors.ctan.org/macros/latex/contrib/fithesis/fithesis/guide/mu/fi.pdf>.
]{fithesis3}

%% The following section sets up the locales used in the thesis.
%% We need to load the T2A font encoding
\usepackage[resetfonts]{cmap}
%% to use the Cyrillic fonts with Russian texts.
\usepackage[T1,T2A]{fontenc}

%% By using `czech` or `slovak` as the main locale instead of `english`,
%% you can typeset the thesis in either Czech or Slovak, respectively.
%% The additional keys allow foreign texts to be typeset as follows:
\usepackage[%
  main=english,
  %% The additional keys allow foreign texts to be typeset as follows:
  % czech, slovak
]{babel}

\usepackage{xcolor}
% \newcommand{\todo}[1]{\textcolor{red}{\textbf{#1}}}

%% The following section sets up the metadata of the thesis.
\thesissetup{%
    date          = \the\year/\the\month/\the\day,
    university    = mu,
    faculty       = fi,
    type          = bc,
    author        = Filip Procházka,
    gender        = m,
    advisor       = RNDr. Jaroslav Bayer,
    title         = {Refactoring of Kdyby packages},
    TeXtitle      = {Refactoring of Kdyby packages},
    keywords      = {package, kdyby, nette, doctrine, orm, composer, packagist},
    TeXkeywords   = {package, kdyby, nette, doctrine, orm, composer, packagist},
}
\thesislong{abstract}{%
    This is the abstract of my thesis, which can

    span multiple paragraphs.
}
\thesislong{thanks}{%
    This is the acknowledgement for my thesis, which can

    span multiple paragraphs.
}

\usepackage[utf8]{inputenc}

%% When typesetting the bibliography, the `numeric` style will be used for the entries
%% and the `numeric-comp` style for the references to the entries.
%% The entries will be sorted in cite order.
%% For more unformation about the available `style`s and `citestyles`, see:
%% <http://mirrors.ctan.org/macros/latex/contrib/biblatex/doc/biblatex.pdf>.
\usepackage{csquotes}
\usepackage[%
  style=numeric,
  citestyle=numeric-comp,
  sorting=none,
  sortlocale=auto,
  block=ragged
]{biblatex}
%% The bibliograpic database
\addbibresource{src/citations.bib}

%% The glossary definitions
\usepackage[acronym,xindy]{glossaries}
\makeglossaries
\loadglsentries[main]{src/glossary.tex}

%% The `makeidx` package contains helper commands for index typesetting.
\usepackage{makeidx}
\makeindex

%% These additional packages are used within the document:
\usepackage{paralist}
\usepackage{amsmath}
\usepackage{amsthm}
\usepackage{amsfonts}
\usepackage{mwe}
\usepackage{graphicx}
\usepackage[iso,english]{isodate}

% csv as table
\usepackage{csvsimple}

% code highlighting
\usepackage{listings}
\usepackage{inconsolata}

% setup higlighting
\definecolor{myPurple}{HTML}{B729D9}
\definecolor{myGreen}{rgb}{0,0.6,0}
\definecolor{myGray}{rgb}{0.5,0.5,0.5}
\definecolor{myMauve}{rgb}{0.58,0,0.82}

% \setmonofont{Courier}
\lstset{ %
  language=php,
  extendedchars = \true,
  keepspaces = true,
  backgroundcolor=\color{white},
  basicstyle=\footnotesize\ttfamily,
  breaklines=true, % automatic line breaking only at whitespace
  captionpos=b, % sets the caption-position to bottom
  commentstyle=\color{gray}, % comment style
  escapeinside={\%*}{*)}, % if you want to add LaTeX within your code
  keywordstyle=\color{blue},       % keyword style
  stringstyle=\color{red},
  morekeywords={%
    abstract, and, array,
    as, break, callable, case, catch, class,
    clone, const, continue, declare, default,
    die, do, echo, else, elseif,
    empty, enddeclare, endfor, endforeach, endif,
    endswitch, endwhile, eval, exit, extends,
    final, finally, for, foreach, function,
    global, goto, if, implements, include,
    include_once, instanceof, insteadof,
    interface, isset, list, namespace,
    new, or, print, private, protected, public,
    require, require_once, return, static,
    switch, throw, trait, try, unset, use, var,
    while, xor, yield,
  }
}

% \lstset{%
%   language = php,
%   basicstyle = \small\ttfamily,
%   extendedchars = \true,
%   commentstyle = \color{gray},
%   keepspaces = true,
%   keywordstyle = \bfseries,
%   % keywordstyle = \color{dkblue},
%   % stringstyle = \color{red},
%   % identifierstyle = \color{dkgreen},
%   % emph =[1]{php},
%   % emphstyle =[1]\color{black},
%   % emph =[2]{if,and,or,else},
%   % emphstyle =[2]\color{dkyellow}
% }
% \lstMakeShortInline[columns=fixed]|

% links setup
\PassOptionsToPackage{hyphens}{url}
\usepackage[%
  % pdftitle={Mongolia games paper},
  % pdfauthor={Daniel Rubenson, Peter Loewen \& Richard Sawyer},
  colorlinks=true,
  urlcolor=blue,
  linkcolor=black,
  anchorcolor=black,
  menucolor=black,
  filecolor=black,
  raiselinks=false,
  hyperfootnotes=true
]{hyperref}

% has to be loaded after hyperref
\usepackage{menukeys}

% fixes line breaks in urls
\makeatletter
\g@addto@macro{\UrlBreaks}{\UrlOrds}
\makeatother

% custom commands
\newcommand{\fnurl}[2]{\href{#2}{#1}\footnote{\url{#2}}}
\newcommand{\hiddensubsection}[1]{
    \stepcounter{subsection}
    \subsection*{\makebox[3em][l]{\arabic{chapter}.\arabic{section}.\arabic{subsection}}{#1}}
}
\newcommand{\hiddensection}[1]{
    \stepcounter{section}
    \section*{\makebox[3em][l]{\arabic{chapter}.\arabic{section}}{#1}}
}

\begin{document}

\chapter{Introduction}

Kdyby is an \gls{oss}~\cite{wiki:oss} project that I, Fi\-li\-p P\-ro\-chá\-z\-ka, lead and maintain. It is a set of PHP~\cite{wiki:php} libraries, that aim to ease writing of web applications.

Through my carer, I have used Kdyby in core business applications of companies such as Damejidlo.cz and Rohlik.cz. A lot of people consider my work useful enough to incorporate it to their own applications as well.

As of writing this, the more popular libraries have hundreds of thousands of downloads. Five of Kdyby libraries have over quarter million downloads and one is approaching half a million with staggering amount of 470 thousands of downloads~\cite{packagist:kdyby}. In conclusion, a sober estimate would be, that Kdyby libraries are used in hundreds of real production applications.

If I account only for the two biggest projects that I can confirm are using Kdyby packages, over a billion Czech crowns\footnote{Rohlik.cz loni prodal zboží za miliardu, letos chce konečně zisk \\\url{http://tyinternety.cz/e-commerce/rohlik-cz-loni-dosahl-na-miliardovy-obrat-letos-chce-konecne-zisk/}} has literary flowed through Kdyby. That is a big responsibility.

Over the years, I have had problems keeping up with the demand and the packages began to get obsolete. I want to use this thesis as way to fix the situation.

I will review the state of each library and decide its future. Which means I will either deprecate it and provide the users a suggestion for a better alternative, or fix the problems and refactor the library.

\chapter{Background for understanding Kdyby}

\section{A brief history of Kdyby} \label{sec:theory:kdyby-history}

In 2006 I have started working on my own \gls{cms}~\cite{wiki:cms}. A prototype was used in production on few websites I created. The oldest preserved version is \fnurl{archived on my \gls{gh}}{https://github.com/fprochazka/kdyby-cms-old}. It is a great learning material on how to not write a CMS.

Then the concept of \gls{oss}~\cite{wiki:oss} was introduced to me and I have decided to start working on everything openly, under a free license~\cite{wiki:fsl}. Mo new working version of Kdyby CMS was released since then, because I have rewritten it from scratch exactly 10 times.

In 2012, I have decomposed the entire system into separate libraries that can be used more or less independently and have their own release cycle. This approach was preserved to this day.

\section{Techniques and design patterns}

\hiddensubsection{Dependency Injection} \label{sec:theory:di}

Inversion of control is a design principle in which custom-written portions of a computer program receive the flow of control from a generic framework.

\Gls{di} is a technique whereby one object supplies the dependencies of another object. Passing the service to the client, rather than allowing a client to build or find the service, is the fundamental requirement of the pattern~\cite{fowler:di}.

\hiddensubsection{Aspect Oriented Programming} \label{sec:theory:aop}

In computing, \gls{aop} is a programming paradigm that aims to increase modularity by allowing the separation of cross-cutting concerns. It does so by adding additional behavior to existing code (an advice) without modifying the code itself, instead separately specifying which code is modified via a pointcut specification, such as log all function calls when the name of the function begins with \lstinline{set}. This allows behaviors that are not central to the business logic (such as logging) to be added to a program without cluttering the code core to the functionality~\cite{wiki:aop}.

\hiddensubsection{Event Dispatcher} \label{sec:theory:event-dispatcher}

The Event Dispatcher is a pattern for writing modular code. It allows to create extension points in the library or application that another library or application can hook into and change or extend the behavior.

Typically, the extension points are called hooks or events and the new functionality is provided with objects called listeners.

\section{Technologies used}

\hiddensubsection{Git and \gls{gh}} \label{sec:theory:git}

Git is a Version Control System that is decentralized and considered very fast~\cite{progit}. \gls{gh} is a collaboration platform for software development using Git.

Each project has a page on \gls{gh} called a repository that can be used to inspect the Git history, files and other metadata. On the project repository page, there are issues and pull requests. Pull requests are a way to ask the maintainer of the repository to incorporate provided code patch to the repository. It can be a bugfix or feature.

There are tools around pull requests that allow collaboration, a code review and discussion about the provided code, so that the maintainers can help the contributors to provide the best code possible.

Kdyby is hosted and developed on \gls{gh}, with the help of several other maintainers and the community that contributes bugfixes and features.

\hiddensubsection{Continuous Integration} \label{sec:theory:ci}

\gls{ci} is a practice of merging all developer working copies to a shared mainline several times a day, to prevent merging conflicts.~\cite{wiki:ci} But nowadays, the term has established to mean CI servers that run prepared task on the provided code.

In practice, it means that as the developer is working on a feature or bugfix, they push the work in progress code into a repository, the code is then picked up by a CI server that executes the tests, checks coding style and runs various other tasks to verify that the code was not broken.

When the work is finished and all the task on CI server completed with success, the code can be probably safely integrated, providing that the tests for new or changed functionality were added.

Some popular CI services are \fnurl{Travis~CI}{https://travis-ci.org/}, \fnurl{CircleCI}{https://circleci.com/} and \fnurl{GitLab~CI}{https://about.gitlab.com/features/gitlab-ci-cd/}. Kdyby is using the Travis~CI that is free for \gls{oss} projects.

\hiddensubsection{Semantic Versioning} \label{sec:theory:semver}

\gls{semver} is a standard that defines how software should be versioned in order to allow safe upgrading. Application might be written with a specific release of the library and upgrading to the newest version might break it because of dropped compatibility. \fnurl{\gls{semver}}{http://semver.org/} defines major, minor and patch releases that signal what versions are compatible with each other.

\hiddensubsection{Nette Framework} \label{sec:theory:nette}

\gls{nette} is an \gls{oss} framework for creating web applications in PHP~\cite{wiki:nette}. \gls{nette} is separated into many packages.

The \gls{di} component \fnurl{nette/di}{https://github.com/nette/di} provides a \gls{dic} that holds the services. The component also allows to prepare a preconfigured \gls{dic}, which is then compiled into a PHP class that contains optimized code for the service creation. This compiled \gls{dic} class is cached for reuse. There is a concept (and a class) CompilerExtension that allows the developer to hook into the process of configuring and compiling of the \gls{nette} \gls{dic}. All of Kdyby packages that are an integration of some other library or tool into \gls{nette} provide a CompilerExtension to make the installation easy.

Default settings for error handling in PHP is not ideal. It outputs errors and warnings directly into browser for the user to see which is bad for esthetical, usability and security reasons. \fnurl{Tracy}{https://github.com/nette/tracy} provides error and exception handlers that redirect the messages into log files. It also renders a red BlueScreen HTML file with extensive details about the error that occurred. The BlueScreen renderer is extensible and allows installing custom panels that can provide additional context for the programmer.

\hiddensubsection{Composer} \label{sec:theory:composer}

Composer is a tool for dependency management~\cite{wiki:package-manager} in PHP. It allows you to declare the libraries your project depends on and it will manage (install or update) them for you~\cite{composer:docs:intro}.

Packages are usually published using \gls{gh} with metadata in a file named \lstinline{composer.json} that is written in JSON~\cite{wiki:json} format.

Composer is decentralized, but has a single main metadata repository \fnurl{Packagist}{https://packagist.org/}. It stores and provides all the package metadata like available versions and where to download them.

All Kdyby libraries are published as Composer packages on Packagist and installing them using the Composer is the only officially supported installation method.

\hiddensubsection{OAuth 2} \label{sec:theory:oauth2}

OAuth is a protocol for authentication and authorization that can be implemented into a web service. It is designed for secure exchange of user information, allowing third party websites to implement a login and registration process that simplifies these tasks for the user essentially allowing them to login or register to services through the OAuth 2 provider with two clicks.

Kdyby provides packages for integrating \gls{nette} with OAuth 2 providers, such as \fnurl{Facebook}{https://developers.facebook.com/docs/facebook-login/manually-build-a-login-flow}, \fnurl{Google}{https://developers.google.com/identity/protocols/OAuth2} and \fnurl{\gls{gh}}{https://developer.github.com/v3/oauth/}.

\hiddensubsection{dibi} \label{sec:theory:dibi}

Dibi is a Database Abstraction Library for PHP. It supports a lot of significant databases: MySQL, PostgreSQL, SQLite, MS SQL, Oracle, Access and generic PDO and ODBC~\cite{dibi:homepage}.

\hiddensubsection{Doctrine 2 ORM} \label{sec:theory:doctrine}

\gls{doctrine} is an \gls{orm}~\cite{wiki:orm}, which means it allows the programmer to create PHP classes called entities that represent relational data in a database and are used to actually map the data from the database to the classes and back. In conclusion, it allows the programmer to write fully \gls{oo}~\cite{wiki:oop} applications.

\hiddensubsection{Symfony Framework} \label{sec:theory:symfony}

\gls{sf} is a PHP web application framework and a set of reusable PHP components/libraries, similar to \gls{nette}~\cite{wiki:symfony}.

For extensibility, \gls{sf} has Bundles that provide similar functionality to \gls{nette} CompilerExtension, but they operate on a different level of abstractions. Bundle is a whole package that contains the Bundle definition, \gls{dic} extension for configuring the Bundle and it may contain adapter classes. Bundles are registered in the AppKernel of Symfony application.

\hiddensubsection{RabbitMQ} \label{sec:theory:rabbitmq}

RabbitMQ is \gls{oss} message broker software (sometimes called message-oriented middleware) that implements the \gls{amqp}. The RabbitMQ server is written in the Erlang programming language and is built on the Open Telecom Platform framework for clustering and failover~\cite{wiki:rabbitmq}.

\hiddensubsection{ElasticSearch} \label{sec:theory:elasticsearch}

\gls{elastic} is a search engine based on Lucene. It provides a distributed, multitenant-capable full-text search engine with an HTTP web interface and schema-free JSON documents. \gls{elastic} is developed in Java and is released as open source under the terms of the Apache License. It is the most popular enterprise search engine~\cite{wiki:elasticsearch}.

\hiddensubsection{Redis} \label{sec:theory:redis}

Redis is an in-memory database \gls{oss} project that is networked, in-memory, and stores keys with optional durability~\cite{wiki:redis}.

\hiddensubsection{PhpStan} \label{sec:theory:phpstan}

PHPStan focuses on finding errors in your code without actually running it. It catches whole classes of bugs even before you write tests for the code~\cite{github:phpstan}.

\hiddensubsection{PHP Standards Recommendations} \label{sec:theory:psr}

The \gls{fig}, which is a group of representatives from established \gls{oss} projects, discusses and creates \gls{psr}. The goal is to discover commonalities in libraries that solve similar problems. The PHP ecosystem is fragmented around tens of frameworks and libraries that all do the same, but slightly differently. This is hugely caused by the absence of a good dependency management tool like Composer, which is still very young. The \gls{psr} contain interfaces that were agreed upon for the libraries to implement. The accepted standards are listed at the \fnurl{\gls{fig} website}{http://www.php-fig.org/psr/}~\cite{fig:psr}.

\hiddensubsection{Monolog} \label{sec:theory:monolog}

Monolog is a logging library that sends your logs to files, sockets, inboxes, databases and various web services. This library implements the PSR-3 logging interface that can be typehinted against in other libraries to keep maximum of interoperability~\cite{monolog:readme}.

\hiddensubsection{Nette\textbackslash{}Tester} \label{sec:theory:nette-tester}

Nette\textbackslash{}Tester is a unit testing framework for PHP~\cite{tester:docs}. Its main advantage over other unit testing libraries is that by default every test runs in single process and the tests are executed in parallel making them run faster, in better isolation and they cannot depend on specific order of execution.

\chapter{Current state of the Kdyby}

To be able to lay out the roadmap, first we have to know the current state of each Kdyby package, the original purpose and the current requirements. We shall only review those packages, that actually made it to production and at least one usable version was released.

Few years back I was really eager to solve all the problems around developing web applications in PHP and I have created few GitHub repositories as a reminder for me, to start working on those problems. And I have actually started to work on some, for example DoctrineForms is one of them, but it was never "officially released". The rest I have not even started working on.

\section{State of the project}

As of 28.4.2017, there are still 68 open pull requests with 622 of them resolved, and 217 open issues, with of them 401 resolved. There is no coding standard being enforced automatically on any package. No static analysis tool is checking the code. But most of the packages have unit and integration tests and linter checking the code for multiple versions of PHP.

Almost all of the packages try to be compatible with PHP 5.4, but \fnurl{PHP 5.4 had end of life at 3.9.2015}{http://php.net/eol.php} and is no longer supported by PHP developers.

\section{State of each package}

This section reviews each package separately, considers the original purpose and sums up the current state.

\hiddensubsection{Doctrine} \label{sec:state:doctrine}

\gls{kDoctrine} is an integration of \gls{doctrine} into Nette Framework.

\gls{doctrine} itself is separated into several packages, mainly \fnurl{doctrine/orm}{https://github.com/doctrine/doctrine2}, \fnurl{doctrine/common}{https://github.com/doctrine/common}, \fnurl{doctrine/annotations}{https://github.com/doctrine/annotations}, \fnurl{doctrine/cache}{https://github.com/doctrine/cache} and \fnurl{doctrine/collections}{https://github.com/doctrine/collections}. What started as a monolith integration in Kdyby, got separated into \gls{kEvents}~\ref{sec:state:events}, \gls{kConsole}~\ref{sec:state:console}, \gls{kAnnotations}~\ref{sec:state:annotations} and \gls{kDoctrineCache}~\ref{sec:state:doctrine-cache} for reusability.

Over the years, it cumulated a lot of responsibilities, that don't belong to it. I have already started extracting few of them in the past, for example an entity prototyping tool~\ref{sec:state:doctrine-magic-accessors}, collection utilities~\ref{sec:state:doctrine-collections-lazy}, \ref{sec:state:doctrine-collections-readonly} and helper for loading big SQL scripts to the database~\ref{sec:state:doctrine-dbal-batch-import}.

There is a big issue \fnurl{Chop up the package}{https://github.com/Kdyby/Doctrine/issues/238} that discusses what other parts should be separated and dropped completely.

New versions of Nette and \gls{doctrine} were released and completely new versions are being prepared, which the integration cannot be currently used with.

\hiddensubsection{Console} \label{sec:state:console}

\gls{kConsole} is an integration of Symfony Framework \lstinline{Console} Component, that allows for writing interactive cli applications. \gls{kDoctrine}~\ref{sec:state:doctrine} depends on this package and is the reason this package exists.

There are tasks, that are better suited for console interaction, than a web interface. Among others, \gls{doctrine} has tools for generating a database schema from the entities metadata and there is a console command for it, that is written using Symfony \lstinline{Console}.

\hiddensubsection{Events} \label{sec:state:events}

\gls{kEvents} provides an event dispatcher~\ref{sec:theory:event-dispatcher} implementation for Nette Framework.

It started as an integration of \gls{doctrine} \lstinline{EventManager}, but then it evolved into a standalone system with support for lazy initialization of listeners and it also contains a naive bridge for Symfony Framework \lstinline{EventDispatcher} Component.

Creating such interchangeable eventing system turned out to be a mistake, because it is a maintenance hell. The systems should have stayed separate.

\hiddensubsection{Annotations} \label{sec:state:annotations}

\gls{kAnnotations} is a simple integration of doctrine/annotations into Nette Framework. It exists solely for the purposes of \gls{kDoctrine}.

\hiddensubsection{DoctrineCache} \label{sec:state:doctrine-cache}

Lorem ipsum.

\hiddensubsection{DoctrineMagicAccessors} \label{sec:state:doctrine-magic-accessors}

Lorem ipsum.

\hiddensubsection{DoctrineCollectionsReadonly} \label{sec:state:doctrine-collections-readonly}

Lorem ipsum.

\hiddensubsection{DoctrineCollectionsLazy} \label{sec:state:doctrine-collections-lazy}

Lorem ipsum.

\hiddensubsection{DoctrineDbalBatchImport} \label{sec:state:doctrine-dbal-batch-import}

Lorem ipsum.

\hiddensubsection{DoctrineForms} \label{sec:state:doctrine-forms}

Lorem ipsum.

\hiddensubsection{Curl} \label{sec:state:curl}

Lorem ipsum.

\hiddensubsection{CurlCaBundle} \label{sec:state:curl-ca-bundle}

Lorem ipsum.

\hiddensubsection{Autowired} \label{sec:state:autowired}

Lorem ipsum.

\hiddensubsection{FormsReplicator} \label{sec:state:forms-replicator}

Lorem ipsum.

\hiddensubsection{Translation} \label{sec:state:translation}

Lorem ipsum.

\hiddensubsection{Validator} \label{sec:state:validator}

Lorem ipsum.

\hiddensubsection{RabbitMq} \label{sec:state:rabbit-mq}

Lorem ipsum.

\hiddensubsection{Money} \label{sec:state:money}

Lorem ipsum.

\hiddensubsection{DoctrineMoney} \label{sec:state:doctrine-money}

Lorem ipsum.

\hiddensubsection{Aop} \label{sec:state:aop}

Lorem ipsum.

\hiddensubsection{Clock} \label{sec:state:clock}

Lorem ipsum.

\hiddensubsection{Redis} \label{sec:state:redis}

Lorem ipsum.

\hiddensubsection{ParseUseStatements} \label{sec:state:parse-use-statements}

Lorem ipsum.

\hiddensubsection{RedisActiveLock} \label{sec:state:redis-active-lock}

Lorem ipsum.

\hiddensubsection{TesterParallelStress} \label{sec:state:tester-parallel-stress}

Lorem ipsum.

\hiddensubsection{Monolog} \label{sec:state:monolog}

Lorem ipsum.

\hiddensubsection{ElasticSearch} \label{sec:state:elastic-search}

Lorem ipsum.

\hiddensubsection{DoctrineSearch} \label{sec:state:doctrine-search}

Lorem ipsum.

\hiddensubsection{Geocoder} \label{sec:state:geocoder}

Lorem ipsum.

\hiddensubsection{CsobPaygateNette} \label{sec:state:csob-paygate-nette}

Lorem ipsum.

\hiddensubsection{CsobPaymentGateway} \label{sec:state:csob-payment-gateway}

Lorem ipsum.

\hiddensubsection{Wkhtmltopdf} \label{sec:state:wkhtmltopdf}

Lorem ipsum.

\hiddensubsection{FakeSession} \label{sec:state:fake-session}

Lorem ipsum.

\hiddensubsection{RequestStack} \label{sec:state:request-stack}

Lorem ipsum.

\hiddensubsection{StrictObjects} \label{sec:state:strict-objects}

Lorem ipsum.

\hiddensubsection{Facebook} \label{sec:state:facebook}

Lorem ipsum.

\hiddensubsection{Google} \label{sec:state:google}

Lorem ipsum.

\hiddensubsection{Github} \label{sec:state:github}

Lorem ipsum.

\hiddensubsection{NettePhpServer} \label{sec:state:nette-php-server}

Lorem ipsum.

\hiddensubsection{TesterExtras} \label{sec:state:tester-extras}

Lorem ipsum.

\hiddensubsection{HtmlValidatorPanel} \label{sec:state:html-validator-panel}

Lorem ipsum.

\hiddensubsection{BootstrapFormRenderer} \label{sec:state:bootstrap-form-renderer}

Lorem ipsum.

\hiddensubsection{PayPalExpress} \label{sec:state:paypal-express}

Lorem ipsum.

\hiddensubsection{PresentersLocator} \label{sec:state:presenters-locator}

Lorem ipsum.

\hiddensubsection{SvgRenderer} \label{sec:state:svg-renderer}

Lorem ipsum.

\hiddensubsection{QrEncode} \label{sec:state:qr-encode}

Lorem ipsum.
\chapter{Roadmap of refactoring} \label{sec:roadmap}

This chapter contains an overview of the plan for refactoring. \ref{sec:roadmap:deprecations} explains what packages will be deprecated and why. \ref{sec:roadmap:common} evaluates standard requirements for all the packages that will undergo refactoring. Finally \ref{sec:roadmap:each-package} describes a specific plan for the individual packages if there are any.

The roadmap extends the scope of this thesis and documents even some changes that are planed in the future.

\section{Deprecations} \label{sec:roadmap:deprecations}

Deprecating a package means it will be visibly marked on \gls{gh} as not maintained anymore and on Packagist there is a particular feature for abandoning packages. When somebody tries to install abandoned the package, Composer will show a warning that the package is not maintained and they should migrate away from it. Deprecation is reversible, and if somebody who is using the package wants to start taking care of it, I will allow it and assign them the maintainer permissions.

New maintainer for \textit{\gls{kCsobPaymentGateway}} and \textit{\gls{kCsobPaygateNette}} was not found and therefore it would be dangerous to encourage the use of these packages in such critical piece of infrastructure without having a maintainer that uses them in production. They are both going to be deprecated. They are well tested and work correctly with the versions of ČSOB payment gateway and \gls{nette} they are written for. The users of these packages will have to migrate to \fnurl{slevomat/csob-gateway}{https://github.com/slevomat/csob-gateway}.

\textit{\gls{kDoctrineSearch}} will be deprecated, and I will make no attempts at fixing it. It is based on a prototype package, and no stable versions were released. If anybody was using it, they were doing so knowingly risking this outcome.

There will be new releases for \gls{nette} 2.3 and 2.4 of \textit{\gls{kFacebook}}, \textit{\gls{kGoogle}} and \textit{\gls{kGithub}}. But after that, the packages will be deprecated. The OAuth 2 is a standard, and all the providers adhere to it with only slight variations. A generic library like \fnurl{league/oauth2-client}{https://github.com/thephpleague/oauth2-client} that implements the standard rather than separate integrations with each provider is more sustainable.

All Kdyby packages will drop support for \gls{hhvm}. It has too low adoption to outweigh the extra maintenance work it requires and supporting \gls{hhvm} on packages that depend on \gls{nette} is impossible.

\section{Common requirements} \label{sec:roadmap:common}

Every package is specific, but there is set of standards that have to be enforced for all good \gls{oss} PHP projects, and it is constantly evolving. Having only tests and documentation is no longer the only best practice. The following sections cover the current trends.

\hiddensubsection{Static analysis with \gls{phpstan}}

\fnurl{\gls{phpstan}}{https://github.com/phpstan/phpstan} claims to be a static analysis tool that can discover bugs in source code. It does not find all bugs, but it has a lot of necessary checks. Running such tool on Kdyby packages in \gls{ci} is going to increase confidence in the correctness of the packages and help prevent introducing new bugs.

But there is a problem with \gls{phpstan} -- it is not a static analysis tool. It does analyze the code, but not statically. It uses two tools for its analysis -- \fnurl{PHP Parser}{https://github.com/nikic/PHP-Parser} and \fnurl{native PHP reflection}{https://secure.php.net/manual/en/book.reflection.php}. First, it parses the source code using the PHP Parser, and when it finds a class, \fnurl{\gls{phpstan} loads the file into memory and executes it}{https://github.com/phpstan/phpstan/issues/137} which makes it available for the PHP reflection. This would not be a problem on itself with source code that has no side effects, but \gls{phpstan} has 3rd party dependencies that might clash with dependencies of the project it is analyzing. One of them is \gls{sfConsole}. If \gls{phpstan} were to analyze the source code of \gls{sfConsole} it would not be analyzing the source code of the library it would be executed on, but the source code of its dependency, because it uses the native PHP reflection.

This problem has several solutions. \gls{phpstan} can be rewritten not to use the native PHP reflection but emulated one that works on top of the PHP Parser. The author does not like this solution because he is worried about the speed of the tool. \gls{phpstan} can be rewritten to drop all dependencies and re--implement the functionality the libraries provide. This solution is not good because it would create additional and unnecessary overhead for the maintainers. Or we can implement a compiler that preprocesses the source code of \gls{phpstan} and its dependencies and fixes the problem.

Citing the PHP documentation, PHAR provides a way to put entire PHP applications into a single file called a PHP Archive for easy distribution and installation~\cite{php:phar}. \gls{phpstan} has also opened an issue \fnurl{PHAR file for each release}{https://github.com/phpstan/phpstan/issues/110} where community is requesting releases to be also made in PHAR.

I am going to implement a compiler that fixes the problem with type collisions and creates a PHAR distribution of the tool. I am also going to offer the author to take over the project afterward so he can make it an official part of the \gls{phpstan} ecosystem.

\hiddensubsection{Coding Standard with \gls{phpCs}}

Kdyby has a coding standard from the beginning that is based on \gls{nette} coding standard, but no tool is automatically enforcing it. I have refused to use \fnurl{\gls{phpCs}}{https://github.com/squizlabs/PHP_CodeSniffer} in the past because it does not have good architecture and did not support the rules \acrlong{kcs} required and somebody would have to implement them first. Now there is \fnurl{slevomat/coding-standard}{https://github.com/slevomat/coding-standard} project that covers most of the needs Kdyby has, and it is reasonable to revisit \gls{phpCs} now.

Kdyby will use \fnurl{consistence/coding-standard}{https://github.com/consistence/coding-standard} as a base definition. \acrlong{ccs} includes the slevomat/coding-standard rules. \acrlong{kcs} will inherit it and modify the rules settings to account for the differences in the standard.

\hiddensubsection{\gls{nette} 2.3 and 2.4 compatible versions}

Each supported package that depends on \gls{nette} must be fixed for \gls{nette} 2.3 if that is not an unreasonable amount of extra work considering the 2.3 version is a legacy now. Otherwise, the 2.3 will be skipped. Then the minimum required version will be increased to \gls{nette} 2.4 and another fixed version will be released that will preferably drop all code that handled backward compatibility with old \gls{nette}. This will allow for less source code and will serve as a better base for future releases.

Releasing the versions that solve compatibility with current \gls{nette} versions is a more important than applying the new Coding Standard and will be prioritized.

\hiddensubsection{PHP 5.6 and newer only}

After the bugfix versions are released, all packages will drop compatibility with PHP 5.5 or older in master branch. No new feature releases will support old PHP unless there is a critical bug that will require a patch release for an older version of the package. Compatibility with PHP 7.0 and 7.1 will be fixed and enforced by \gls{ci}.

All the packages have gathered some bug reports in their issue trackers. What can be fixed for \gls{nette} 2.3 or 2.4 will be fixed before that release. Everything else that requires architecture changes to fix the problems or implement new features will not be implemented before the minimum requirement of PHP 5.6 is enforced. It will also not be implemented before the package is fixed based on the \gls{phpstan} analysis and new Coding Standard is applied and enforced.

\hiddensubsection{PHP 7.1 and newer}

Kdyby packages will skip minimum requirement of PHP 7.0. After the support for PHP 5.6 is dropped the support for PHP 7.0 will be dropped with it. PHP 7.0 introduces return value typehinting and scalar typehinting, allowing to declare if argument should be string or integer which was not possible until PHP 7.0. But it is missing \fnurl{nullable types}{https://secure.php.net/manual/en/migration71.new-features.php\#migration71.new-features.nullable-types} and \fnurl{void return type}{https://secure.php.net/manual/en/migration71.new-features.php\#migration71.new-features.void-functions} which are both important and the new type system is incomplete without them.

PHP 7.1 releases will not be part of refactoring covered by this thesis, but they are an essential element of the roadmap and should be mentioned.

\hiddensubsection{Framework agnostic libraries}

Most of the packages are an integration of some tool into \gls{nette}, but many of them extend the functionality of the original package making them a candidate for a split into two or more packages. An excellent example of this is \textit{\gls{kDoctrine}} that accumulated many extra features. If such package were installed into a \gls{sf} application, it would drag along all its dependencies and tight coupling for \gls{nette}. The additional dependencies can be ignored, and it would most likely work, but that is not the best way to develop applications.

The solution to this problem is to extract the functionality from the packages that violate the Unix philosophy to do one thing only and to do it right. The extracted packages that are framework agnostic can be used with \gls{sf} or with other PHP frameworks easily, and the \gls{nette} community will benefit from bigger potential user base which inherently makes any \gls{oss} better.

\section{Specific requirements for each package} \label{sec:roadmap:each-package}

This section will only broadly cover the interesting and significant architectural changes that will be made and not go in depth on all the issues that have to be fixed.

\hiddensubsection{Console}

The introduction of \lstinline{LinkGenerator} service in \gls{nette} resolved the problem with generating URLs. \textit{\gls{kConsole}} can, therefore, drop the entire abstraction that was fixing the issue.

\hiddensubsection{Events}

After fixing the compatibility with \gls{nette} 2.3 and 2.4, there will be a significant change of philosophy in this package. The EventManager that \textit{\gls{kDoctrine}} requires will be simplified and moved to a bridge package between \gls{nette} and \gls{doctrine}. Its only responsibility will be making sure the event subscriber classes for \gls{doctrine} are lazy--initialized and fetched from \gls{dic} only when they are required. The rest of the package will be deprecated, and users will be encouraged to use \gls{sfEventDispatcher} in new projects.

% \hiddensubsection{DoctrineDbalBatchImport}

% This package has no stable release yet because it is only one helper class extracted from \textit{\gls{kDoctrine}}. It will be refactored to a modern \gls{oo} API.

\hiddensubsection{Doctrine}

\gls{doctrine} dropped support for older PHP in \lstinline{master} branches and will have the newest releases only for PHP 7.1 and newer. After the Nette 2.3 and 2.4 versions with bugfixes are released, the master will switch the minimum dependency on PHP to 7.1 and the refactored version will be written directly in PHP 7.1 and newer.

\textit{\gls{kDoctrine}} provides a Tracy panel that violates \gls{srp} because it solves rendering SQL queries, rendering second level cache statistics and rendering of Tracy BlueScreen panels for exceptions. The SQL Panel for Tracy will be separated to standalone package \textit{\gls{kTracyDoctrineDbalPanel}} so it can be used with only Tracy and Doctrine with no other dependencies required. The remaining functionality will be extracted into two different classes where one renders cache statistics in Tracy panel and the other render panels with additional context for Tracy BlueScreens.

Manager class from \gls{doctrine} has been extended to provide extra configuration options and custom diagnostics features. But \gls{doctrine} internally assumes the \lstinline{EntityManager} is final and should not be extended, only decorated. The inherited class can be removed because the problems it solved can be resolved without inheritance.

\begin{sloppypar}
The structure of configuration will be changed to be roughly the same as \gls{sf} Doctrine Bundle has. Together with that, the \lstinline{CompilerExtension} will be refactored to only register services that are necessary for \gls{doctrine} to run correctly.
\end{sloppypar}

The \lstinline{EntityManager} holds state and has a \lstinline{close} feature that when exception rises while it is synchronizing state with the database, it gets locked and will not perform any operations. That is perfectly reasonable, but the application must be able to recover from such state. \lstinline{ManagerRegistry} solves that partially, and \textit{\gls{kDoctrine}} already implements it. But a better solution is to utilize \fnurl{ocramius/proxy-manager}{https://github.com/Ocramius/ProxyManager} that creates a proxy for the manager and allows to reset the service completely, even when other services have already reference for the services.

A lifecycle event subscriber for a particular entity type is called entity listener. The entity listeners often have dependencies that should be injected using \gls{di}. That will be achieved with a custom implementation of entity listener resolver \lstinline{ContainerAwareEntityListenerResolver} that will fetch the configured listeners from \gls{dic}.

A \lstinline{ConnectionFactory} will be implemented to handle registration of custom data types for the entity fields. This is now solved directly in inherited \lstinline{Doctrine\DBAL\Connection} but that violates \gls{srp}.

Custom \lstinline{ContainerAwareEventManager} for lazy initialization of subscriber classes will be a part of \textit{\gls{kDoctrine}} directly, and dependency on \textit{\gls{kEvents}} will be removed.

To be able to configure parameters for \gls{doctrine} filters a \lstinline{ManagerConfigurator} will be implemented that will hold the parameters and pass them into the filters.

The \lstinline{NonLockingUniqueInserter} for atomic inserts can be extracted into a standalone package with dependency only on \gls{doctrine}.

Single package will be extracted with \lstinline{QueryObject} and \lstinline{ResultSet} for writing self--contained \gls{dql} queries.

Repositories reimplemented using \gls{dql} and extended \lstinline{QueryBuilder} with auto--join feature will be extracted into a standalone package.

All the extracted packages except \textit{\gls{kTracyDoctrineDbalPanel}} should be completely optional and framework agnostic. This will allow using them with other frameworks, not only \gls{nette}.

\hiddensubsection{Translation}

The \lstinline{^3.0} releases of \gls{sfTranslation} solve problems that \\\textit{Kdyby\textbackslash{}Translation} tried to fix, and a lot of code can be removed completely.

\hiddensubsection{Clock}

\textit{\gls{kClock}} will be separated into two packages where one would only cover implementing the \lstinline{DateTimeProvider} itself, and the other package will integrate it with \gls{nette}.

\hiddensubsection{Geocoder}

Four new packages will be extracted from \textit{\gls{kGeocoder}}, and this package will be deprecated. There will be \textit{\gls{kGeocoderLogging}}, \textit{\gls{kGeocoderSeznamMaps}}, \textit{\gls{kGeocoderGoogleMapsProxied}} and \textit{\gls{kGeocoderBestMatch}}.

\hiddensubsection{Wkhtmltopdf}

This package implements a custom process handling that will be replaced with \fnurl{symfony/process}{https://github.com/symfony/process}. It also contains both the implementation of the CLI \gls{oo} abstraction and the \gls{nette} integration itself. These responsibilities will be separated into individual packages.

\chapter{The refactoring process of Kdyby}

This chapter documents what I have accomplished with each package in detail.

\section{Doctrine}

Lorem ipsum.

\section{Console}

Lorem ipsum.

\section{Events}

Lorem ipsum.

\section{Annotations}

Lorem ipsum.

\section{DoctrineCache}

Lorem ipsum.

\section{DoctrineMagicAccessors}

Lorem ipsum.

\section{DoctrineCollectionsReadonly}

Lorem ipsum.

\section{DoctrineCollectionsLazy}

Lorem ipsum.

\section{DoctrineDbalBatchImport}

Lorem ipsum.

\section{DoctrineForms}

Lorem ipsum.

\section{Autowired}

Lorem ipsum.

\section{FormsReplicator}

Lorem ipsum.

\section{Translation}

Lorem ipsum.

\section{Validator}

Lorem ipsum.

\section{RabbitMq}

Lorem ipsum.

\section{Money}

Lorem ipsum.

\section{DoctrineMoney}

Lorem ipsum.

\section{Aop}

Lorem ipsum.

\section{Clock}

Lorem ipsum.

\section{Redis}

Lorem ipsum.

\section{ParseUseStatements}

Lorem ipsum.

\section{RedisActiveLock}

Lorem ipsum.

\section{TesterParallelStress}

Lorem ipsum.

\section{Monolog}

Lorem ipsum.

\section{ElasticSearch}

Lorem ipsum.

\section{DoctrineSearch}

Lorem ipsum.

\section{Geocoder}

Lorem ipsum.

\section{CsobPaygateNette}

Lorem ipsum.

\section{CsobPaymentGateway}

Lorem ipsum.

\section{Wkhtmltopdf}

Lorem ipsum.

\section{FakeSession}

Lorem ipsum.

\section{RequestStack}

Lorem ipsum.

\section{StrictObjects}

Lorem ipsum.

\section{Facebook}

Lorem ipsum.

\section{Google}

Lorem ipsum.

\section{Github}

Lorem ipsum.

\section{NettePhpServer}

Lorem ipsum.

\section{TesterExtras}

Lorem ipsum.

\section{HtmlValidatorPanel}

Lorem ipsum.

\chapter{Conclusion}

The main goal of this thesis was to release as many new versions as possible of Kdyby packages that are compatible with new versions of libraries they integrate with. The goal to start checking the packages with \gls{phpstan} and have a coding standard is important but not critical. And finally separating the packages into smaller packages with less responsibilities has advantages but the biggest one is marketing where the programmers will not feel intimidated by the amount of code in the package and will be more inclined to actually install it and therefore is also not as critical.

Almost all of the most popular packages have now versions that support current Nette, Doctrine and Symfony. And only few of those do not have \gls{phpstan} or \acrlong{kcs} configured on \gls{ci} server. Not all packages are resolved, but that is fine. If the work was rushed it would beat the purpose of refactoring them.

The amount of work covered by this thesis represents several months of work but only a fraction of what have already been invested into Kdyby by the maintainers and community.

\section{Future work}

The roadmap chapter~\ref{sec:roadmap} of this thesis covers the big architectural changes for Kdyby packages and only a part of that plan was implemented. By solving them many of the open issues will be also resolved. As of writing this, there are 188 open issues and pull requests that have to be solved or reviewed.

The issues and pull requests alone are several months of work and the architectural changes will take at least half a year to implement.


\printbibliography

\end{document}
