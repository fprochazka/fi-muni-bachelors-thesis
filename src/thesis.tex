\documentclass[
  color, %% This option enables colorful typesetting. Replace with
         %% `monochrome`, if you are going to print the thesis on
         %% a monochromatic printer.
  table, %% Causes the coloring of tables. Replace with `notable`
         %% to restore plain tables.
  lof,   %% Prints the List of Figures. Replace with `nolof` to
         %% hide the List of Figures.
  lot,   %% Prints the List of Tables. Replace with `nolot` to
         %% hide the List of Tables.
  %% More options are listed in the class documentation at
  %% <http://mirrors.ctan.org/macros/latex/contrib/fithesis/fithesis/guide/mu/fi.pdf>.
]{fithesis3}
%% We need to load the T2A font encoding
\usepackage[resetfonts]{cmap}
%% to use the Cyrillic fonts with Russian texts.
\usepackage[T1,T2A]{fontenc}

%% By using `czech` or `slovak` as the main locale instead of `english`,
%% you can typeset the thesis in either Czech or Slovak, respectively.
%% The additional keys allow foreign texts to be typeset as follows:
\usepackage[
  main=english,
  german, russian, czech, slovak
]{babel}
\usepackage{xcolor}
% \newcommand{\todo}[1]{\textcolor{red}{\textbf{#1}}}
% \usepackage{listings}

%% The following section sets up the metadata of the thesis.
\thesissetup{
    university    = mu,
    faculty       = fi,
    type          = bc,
    author        = Filip Procházka,
    gender        = m,
    advisor       = RNDr. Jaroslav Bayer,
    title         = {Refactoring of Kdyby packages},
    TeXtitle      = {Refactoring of Kdyby packages},
    keywords      = {package, kdyby, nette, doctrine, orm, composer, packagist},
    TeXkeywords   = {package, kdyby, nette, doctrine, orm, composer, packagist},
}
\thesislong{abstract}{
    This is the abstract of my thesis, which can

    span multiple paragraphs.
}
\thesislong{thanks}{
    This is the acknowledgement for my thesis, which can

    span multiple paragraphs.
}

%% When typesetting the bibliography, the `numeric` style will be used for the entries
%% and the `numeric-comp` style for the references to the entries.
%% The entries will be sorted in cite order.
%% For more unformation about the available `style`s and `citestyles`, see:
%% <http://mirrors.ctan.org/macros/latex/contrib/biblatex/doc/biblatex.pdf>.
\usepackage{csquotes}
\usepackage[
  backend=bibtex,
  style=numeric,
  citestyle=numeric-comp,
  sorting=none,
  sortlocale=auto
]{biblatex}

%% The bibliograpic database
\addbibresource{citations.bib}

%% The `makeidx` package contains helper commands for index typesetting.
\usepackage{makeidx}
\makeindex

%% These additional packages are used within the document:
\usepackage{paralist}
\usepackage{amsmath}
\usepackage{amsthm}
\usepackage{amsfonts}
\usepackage{url}
\usepackage{menukeys}
\usepackage{mwe}

\begin{document}

\chapter{One chapter}
\section{First}
Lorem ipsum dolor sit amet, consectetuer adipiscing elit. Mauris tincidunt sem sed arcu. Nulla pulvinar eleifend sem. In enim a arcu imperdiet malesuada. Nunc dapibus tortor vel mi dapibus sollicitudin. Pellentesque pretium lectus id turpis. Pellentesque arcu. Nemo enim ipsam voluptatem quia voluptas sit aspernatur aut odit aut fugit, sed quia consequuntur magni dolores eos qui ratione voluptatem sequi nesciunt. Morbi imperdiet, mauris ac auctor dictum, nisl ligula egestas nulla, et sollicitudin sem purus in lacus. Vivamus ac leo pretium faucibus. Lorem ipsum dolor sit amet, consectetuer adipiscing elit. Ut tempus purus at lorem.

Vivamus ac leo pretium faucibus. In convallis. Aliquam erat volutpat. Mauris metus. Temporibus autem quibusdam et aut officiis debitis aut rerum necessitatibus saepe eveniet ut et voluptates repudiandae sint et molestiae non recusandae. Integer imperdiet lectus quis justo. Fusce wisi. Integer lacinia. Etiam posuere lacus quis dolor. In rutrum. Pellentesque arcu. Temporibus autem quibusdam et aut officiis debitis aut rerum necessitatibus saepe eveniet ut et voluptates repudiandae sint et molestiae non recusandae. Duis sapien nunc, commodo et, interdum suscipit, sollicitudin et, dolor.

\section{Second}
Pellentesque habitant morbi tristique senectus et netus et malesuada fames ac turpis egestas. Phasellus faucibus molestie nisl. Maecenas aliquet accumsan leo. Aliquam ante. Nulla est. Curabitur bibendum justo non orci. Pellentesque arcu. Donec vitae arcu. Nullam eget nisl. Nullam faucibus mi quis velit. Morbi leo mi, nonummy eget tristique non, rhoncus non leo.

Praesent vitae arcu tempor neque lacinia pretium. Duis pulvinar. Excepteur sint occaecat cupidatat non proident, sunt in culpa qui officia deserunt mollit anim id est laborum. Integer malesuada. Aliquam in lorem sit amet leo accumsan lacinia. Duis risus. Morbi imperdiet, mauris ac auctor dictum, nisl ligula egestas nulla, et sollicitudin sem purus in lacus. Pellentesque habitant morbi tristique senectus et netus et malesuada fames ac turpis egestas. Fusce tellus odio, dapibus id fermentum quis, suscipit id erat. Aenean fermentum risus id tortor. Duis condimentum augue id magna semper rutrum. Aliquam in lorem sit amet leo accumsan lacinia. Maecenas sollicitudin. Etiam egestas wisi a erat. Maecenas sollicitudin. In laoreet, magna id viverra tincidunt, sem odio bibendum justo, vel imperdiet sapien wisi sed libero. Quisque tincidunt scelerisque libero. In sem justo, commodo ut, suscipit at, pharetra vitae, orci. Proin in tellus sit amet nibh dignissim sagittis. Mauris dictum facilisis augue.

\chapter{Two chapter}
\section{Third}
Integer lacinia. Nullam dapibus fermentum ipsum. Aliquam ante. Mauris elementum mauris vitae tortor. Cras pede libero, dapibus nec, pretium sit amet, tempor quis. Nulla est. Proin pede metus, vulputate nec, fermentum fringilla, vehicula vitae, justo. Aliquam erat volutpat. Sed ut perspiciatis unde omnis iste natus error sit voluptatem accusantium doloremque laudantium, totam rem aperiam, eaque ipsa quae ab illo inventore veritatis et quasi architecto beatae vitae dicta sunt explicabo. Donec ipsum massa, ullamcorper in, auctor et, scelerisque sed, est. Duis sapien nunc, commodo et, interdum suscipit, sollicitudin et, dolor.

\end{document}
