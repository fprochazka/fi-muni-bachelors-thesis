\chapter{Conclusion}

The main goal of this thesis was to release as many new versions as possible of Kdyby packages that are compatible with new versions of libraries they integrate with. The goal to start checking the packages with \gls{phpstan} and have a coding standard is important but not critical. And finally separating the packages into smaller packages with less responsibilities has advantages but the biggest one is marketing where the programmers will not feel intimidated by the amount of code in the package and will be more inclined to actually install it and therefore is also not as critical.

Almost all of the most popular packages have now versions that support current Nette, Doctrine and Symfony. And only few of those do not have \gls{phpstan} or \acrlong{kcs} configured on \gls{ci} server. Not all packages are resolved, but that is fine. If the work was rushed it would beat the purpose of refactoring them.

The amount of work covered by this thesis represents several months of work but only a fraction of what have already been invested into Kdyby by the maintainers and community.

\section{Future work}

The roadmap chapter~\ref{sec:roadmap} of this thesis covers the big architectural changes for Kdyby packages and only a part of that plan was implemented. By solving them many of the open issues will be also resolved. As of writing this, there are 188 open issues and pull requests that have to be solved or reviewed.

The issues and pull requests alone are several months of work and the architectural changes will take at least half a year to implement.
